\capitulo{7}{Conclusiones y Líneas de trabajo futuras}

En esta sección se exponen las conclusiones derivadas del trabajo, así como las posibles evoluciones futuras de PrimeBot para realizar nuevas versiones del proyecto.

\section{Conclusiones}\label{conclusiones}

Tras el desarrollo del proyecto podemos concluir que:

\begin{itemize}
\tightlist
\item
 El objetivo principal del proyecto se ha cumplido correctamente, llegando a la solución para cada una de las pruebas seleccionadas inicialmente e incluso añadiendo funcionalidades no contempladas en un inicio.
\item
 El uso de una forma de desarrollo ágil como SCRUM y el sistema de Sprints ha permitido realizar un desarrollo ordenado y eficaz aprovechando al máximo los recursos.
\item
 Gracias a la parte de investigación y búsqueda de elementos, proyectos y trabajo realizado con la comunidad Open Source, se ha podido llegar a una solución muy buena para cada una de las pruebas.
\item
 El empleo del ecosistema Arduino y la gran comunidad, además de la cantidad de sensores y elementos compatibles directamente con la plataforma han permitido poder contemplar muchas opciones para cada una de las soluciones que iba a realizar PrimeBot.
\end{itemize}

\section{Trabajo futuro}\label{trabajo-futuro}

A lo largo del desarrollo del proyecto han salido nuevas versiones de Arduino que sería interesante implementar en las versiones posteriores de PrimeBot.

\begin{itemize}
\tightlist
\item
 Incorporación de STM32: las nuevas versiones de arduino han incorporado el microcontrolador STM32 que puede incorporar una gran mejora de rendimiento para este tipo de proyectos, siendo ese microcontrolador uno de los más populares en proyectos de robots siguelíneas.
\item
 Preparación del resto de pruebas: actualmente PrimeBot en su primera versión no realiza todas las pruebas disponibles en el ASTI Robotics Challenge, una nueva línea de implementación sería realizar el código para afrontar el resto de pruebas disponibles.
\end{itemize}
