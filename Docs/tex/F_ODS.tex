\apendice{Anexo de sostenibilización curricular}

\section{Introducción}

El desarrollo de PrimeBot no solo se centró en la creación de un robot eficiente y competitivo, sino que también se consideraron aspectos importantes de sostenibilidad. Este anexo proporciona una reflexión personal sobre las competencias de sostenibilidad adquiridas durante el desarrollo del proyecto y cómo se han aplicado en el Trabajo de Fin de Grado. La sostenibilidad en la ingeniería es crucial para asegurar que las soluciones tecnológicas no solo sean efectivas, sino también responsables con el medio ambiente y la sociedad.

\section{Eleccion de materiales}

Durante el desarrollo de PrimeBot, una de las primeras consideraciones de sostenibilidad fue la elección de materiales y componentes. Se priorizaron componentes electrónicos de bajo consumo energético, como el Arduino Nano y los sensores de Pololu, que son eficientes y reducen el impacto ambiental. Además, se utilizó un PCB personalizado para minimizar el uso de cables y componentes adicionales, reduciendo así los residuos electrónicos.

Además los componentes realizados a través de impresión 3D han sido fabricados con los materiales menos contaminantes que hay en el mercado.

\section{Optimización energética}
El algoritmo PID y otros algoritmos de control fueron diseñados para ser eficientes en términos de consumo de energía. PrimeBot fue programado para optimizar el uso de energía de los motores y sensores, lo que no solo mejora el rendimiento del robot, sino que también extiende su vida útil y reduce la necesidad de recargas frecuentes.

\section{Reciclaje y Reutilización}

Otro aspecto de sostenibilidad abordado en el proyecto fue el reciclaje y la reutilización de componentes. Muchos de los componentes utilizados en PrimeBot fueron reutilizados de proyectos anteriores, reduciendo la necesidad de adquirir nuevos materiales. 

\section{Gestión de Residuos}
La gestión de residuos fue un aspecto clave en todas las fases del proyecto. Durante la fase de prototipado, se implementaron prácticas para minimizar los residuos generados, como la reducción de impresiones y el uso eficiente de materiales durante el montaje del PCB. Los componentes obsoletos o defectuosos fueron desechados de acuerdo con las normativas de gestión de residuos electrónicos.

\section{Conclusión}
El desarrollo de PrimeBot me permitió adquirir y aplicar diversas competencias de sostenibilidad, desde el uso responsable de recursos y la optimización energética, hasta la promoción de la economía circular y la educación social. Estas competencias no solo son relevantes para el proyecto específico de PrimeBot, sino que también serán esenciales en mi futura carrera como ingeniero. La integración de prácticas sostenibles en proyectos de ingeniería es crucial para contribuir a un futuro más sostenible y responsable.