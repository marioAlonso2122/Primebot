\capitulo{4}{Técnicas y herramientas}

\section{Metodologías}\label{metodologias}

\subsection{Scrum}\label{scrum}

Scrum es un marco de trabajo que busca aplicar de manera regular un conjunto de buenas prácticas para el desarrollo de \emph{software}  que se coloca dentro de las metodologías ágiles. En Scrum se realizan entregas parciales y regulares del producto final realizando una estrategia de trabajo iterativa e incremental.

\subsection{Técnica Pomodoro}\label{pomodoro}

La técnica Pomodoro es un método de gestión de tiempo para realizar el trabajo en intervalos incrementando la productividad.
Para aplicar esta técnica se divide la tarea a realizar en intervalos de 45 minutos en mi caso, durante estos 45 minutos se evita cualquier distracción y se descansan 5 minutos al finalizar ese intervalo.
Este periodo de tiempo se conoce como un pomodoro, cada cuatro pomodoros se descansan 30 minutos.

\section{\emph{Hosting} del repositorio}\label{hosting-del-repositorio}

\subsection{GitHub}\label{GitHub}
GitHub es la plataforma web de hospedaje de repositorios más popular en el mundo.
Ofrece todas las funcionalidades de Git y muchas otras integraciones además de ser gratuita.

He utilizado GitHub como plataforma principal donde se va a alojar todo el código del proyecto.

\section{Gestión del proyecto}\label{gestion-del-proyecto}

\subsection{YouTrack}\label{YouTrack}

YouTrack es una herramienta de gestión de proyectos gratuita que permite realizar una implementación SCRUM de forma sencilla.
Gracias a YouTrack podemos trabajar con un tablero canvas y dividir cada sprint en diferentes tareas además de realizar diagramas de Gantt e informes del proyecto.

\section{Entorno de desarrollo integrado
(IDE)}\label{entorno-de-desarrollo-integrado-ide}

\subsection{Arduino IDE}\label{arduino-IDE}

Para el desarrollo del código principal que va incorporado en el microcontrolador de PrimeBot se ha utilizado el IDE Oficial de Arduino.
Este entorno de desarrollo nos aporta todas las herramientas necesarias para llevar a cabo el proyecto, además nos permite trabajar con monitor serial para poder hacer pruebas de conectividad Bluetooth.

\subsection{Visual Studio Code}\label{visual-studio-code}

Visual Studio Code es un editor de código fuente gratuito desarrollado por Microsoft.
Se ha utilizado para realizar el desarrollo de la página web de PrimeBot.

\section{Documentación}\label{documentacion}

\subsection{LaTeX}\label{latex}

LaTeX es un sistema de composición de textos que genera documentos de una gran calidad tipográfica.
Este sistema está muy extendido en el ámbito científico para la generación de artículos y libros. ~\cite{wiki:latex}

TeXShop es un editor de LaTeX y Tex de código abierto para macOS que ha sido utilizado para la redacción de la documentación del proyecto.